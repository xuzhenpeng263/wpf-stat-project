\documentclass[11pt, a4paper]{article}

% --- 中文支持与字体设置 ---
\usepackage{xeCJK}
% 将原有的字体设置替换为:
\setCJKmainfont{SimSun}          % 宋体
\setCJKsansfont{SimHei}          % 黑体
\setCJKmonofont{NSimSun}         % 新宋体(等宽))

% --- 基础宏包 ---
\usepackage{amsmath, amssymb}
\usepackage{graphicx}
\usepackage[a4paper, margin=1in]{geometry}
\usepackage{hyperref}
\usepackage{listings}
\usepackage{xcolor}
\usepackage{caption}
\usepackage{authblk}
\usepackage{booktabs} 
\usepackage{geometry}
\usepackage[nottoc]{tocbibind} % 将目录、参考文献等加入到目录中

% --- 超链接设置 ---
\hypersetup{
    colorlinks=true,
    linkcolor=blue,
    filecolor=magenta,      
    urlcolor=cyan,
    pdftitle={WPF-Stat 中文论文},
    pdfpagemode=FullScreen,
}

% --- 代码块 (Listings) 设置 ---
\definecolor{codegreen}{rgb}{0,0.6,0}
\definecolor{codegray}{rgb}{0.5,0.5,0.5}
\definecolor{codepurple}{rgb}{0.58,0,0.82}
\definecolor{backcolour}{rgb}{0.95,0.95,0.92}

\lstdefinestyle{mystyle}{
    backgroundcolor=\color{backcolour},   
    commentstyle=\color{codegreen},
    keywordstyle=\color{magenta},
    numberstyle=\tiny\color{codegray},
    stringstyle=\color{codepurple},
    basicstyle=\ttfamily\footnotesize,
    breakatwhitespace=false,         
    breaklines=true,                 
    captionpos=b,                    
    keepspaces=true,                 
    numbers=left,                    
    numbersep=5pt,                  
    showspaces=false,                
    showstringspaces=false,
    showtabs=false,                  
    tabsize=2,
    language=Python
}
\lstset{style=mystyle}

% --- 标题与作者 ---
\title{\textbf{WPF-Stat:通过加权质因数分解熵探测随机数生成器的结构完整性}}
\author[1]{徐振鹏}
\affil[1]{北华航天工业学院}
\date{\today}

% --- 文档开始 ---
\begin{document}

\maketitle

\begin{abstract}
\noindent 高质量的随机数生成器 (RNG) 是现代计算的基石,然而,传统的验证套件通常聚焦于比特层面的统计特性,而对更深层次的数论结构保持“盲视”。本文引入了一个新颖的随机性测试框架——\texttt{WPF-Stat},其核心思想是:一个真正随机的整数序列,在结构上应该与自然数序列无法区分。我们采用“加权质因数分解熵”(WPF Entropy)作为度量标准,来量化每个整数内在的结构复杂度。通过将被测序列的WPF熵分布与一个从自然数集中预先计算的基准进行比较,我们为随机性验证建立了一个全新的、正交的维度。我们的实验表明,Python生态系统中的高质量密码学和科学计算RNG(如 \texttt{secrets}, \texttt{numpy.random}, \texttt{random})所产生的序列,与自然基准完美契合。相反,一个简单的线性同余生成器 (LCG) 则在其结构分布和序列相关性上表现出显著且可被探测到的偏差。此外,我们的研究揭示了数字空间中熵的一个基本尺度效应,即特征熵分布会随着所分析数值的量级变化而发生可预测的偏移。我们断定,\texttt{WPF-Stat} 是一个有效且富有洞察力的工具,能够识别出传统方法所忽略的精细结构性缺陷。
\end{abstract}

\clearpage
\tableofcontents
\clearpage

\section{引言}

随机数生成器 (RNG) 是现代科学、密码学和模拟计算的基石。其输出质量至关重要,因为非随机的人为痕迹可能导致科学成果失效或造成灾难性的安全漏洞。因此,学术界和工业界开发了大量的统计测试套件,例如NIST SP 800-22和Dieharder电池测试,以验证RNG的质量。这些套件擅长于对一个比特流进行统计学上的“尸检”,验证其频率、游程、秩等属性。

然而,这些测试在很大程度上对构成比特流的整数的数值意义是不可知的。它们是“结构盲视”的。可以想象,一个生成器可能产生一个通过了所有比特级测试的序列,但却在其数论属性上表现出显著的偏好——例如,对素数或具有简单因子分解的数有微小的过度生产。

本文基于一个核心假设:\textit{一个在区间 $[a, b]$ 上理想的随机整数生成器,其产生的序列在数论属性上,应该与从该区间所有整数中均匀随机抽样的结果在统计上无法区分。} 为了验证这一假设,我们需要一个能够量化整数“结构复杂度”的度量。为此,我们利用了“加权质因-数分解熵”(WPF Entropy)的概念。

我们呈现了 \texttt{WPF-Stat},一个基于Python的测试套件,它实现了这一哲学。它首先通过计算大量自然数的WPF熵分布,建立一个“黄金标准”。然后,它将被测RNG产生的序列与此基准进行一系列统计检验,从而对其质量提供一个新颖的、正交的评估。

\section{理论框架:WPF熵}

我们方法论的核心是加权质因数分解熵 $H(n)$,一个旨在捕获整数 $n$ 结构复杂度的度量。给定整数 $n$ 的标准质因数分解形式 $n = p_1^{k_1} p_2^{k_2} \cdots p_m^{k_m}$,其WPF熵被定义为三个不同分量的乘积:

\begin{equation}
    H(n) = H_{\text{结构}}(n) \cdot \Phi_{\text{阶数}}(n) \cdot W_{\text{权重}}(n)
\end{equation}

\paragraph{1. 结构熵 ($H_{\text{结构}}$)} 该分量基于香农熵公式,用于测量质因子指数的不确定性或“混合度”。令 $A = \sum_{j=1}^{m} k_j$ 为所有指数之和,则每个指数的归一化概率为 $P_i = k_i / A$。

\begin{equation}
    H_{\text{结构}}(n) = - \sum_{i=1}^{m} P_i \log_2 P_i
\end{equation}
对于质数幂 ($n=p^k$, $m=1$),$P_1 = 1$ 且 $H_{\text{结构}}(n) = 0$,这正确地将它们识别为结构上的“纯态”。

\paragraph{2. 阶数乘子 ($\Phi_{\text{阶数}}$)} 该项奖励那些由更多总质因子(计入重数)构成的整数。

\begin{equation}
    \Phi_{\text{阶数}}(n) = 1 + \log_2 A
\end{equation}

\paragraph{3. 平均权重 ($W_{\text{权重}}$)} 该分量赋予由更大质因子构成的整数更高的复杂度,反映了更大质数本身所含的信息量。

\begin{equation}
    W_{\text{权重}}(n) = \frac{1}{m} \sum_{i=1}^{m} \log_2 p_i
\end{equation}

\section{方法论: \texttt{WPF-Stat} 测试套件}

\texttt{WPF-Stat} 工具被实现为一个Python类 `WPFAnalyzer`,它协调了整个分析流程。

\subsection{基准生成}
第一步是建立一个“黄金标准”参考分布。该工具计算从2到指定上限(例如 $10^6$)的每个整数的WPF熵。这是一个计算密集型过程,通过多核并行计算来加速。最终得到的熵值数组,构成了我们在该数值范围内“自然”结构分布的经验模型。

\subsection{统计检验电池}
在为被测序列计算出WPF熵后,我们执行以下统计检验,将其与基准分布进行比较:

\begin{enumerate}
    \item \textbf{分布相似性 (Kolmogorov-Smirnov 检验):} 双样本K-S检验比较被测序列熵值的经验累积分布函数 (ECDF) 与一个同等大小的基准样本的ECDF。它对分布形状、中位数或离散度的任何系统性偏移都很敏感。

    \item \textbf{微观结构 ($\chi^2$ 检验):} 该检验评估不同类型数字的比例是否正确。我们将熵值分箱(为零熵数字设一个专用箱),并使用皮尔逊 $\chi^2$ 检验比较被测序列中的观测频次与从基准中推导出的期望频次。这对于探测对素数的偏好尤其有效。

    \item \textbf{熵聚集性 (Wald-Wolfowitz 游程检验):} 为探测序列相关性,我们对相对于其中位数的熵序列执行游程检验。观测到的游程数与期望值的显著统计偏差,表明相似复杂度的数字倾向于“聚集”在一起,这是非随机性的一个明确信号。
\end{enumerate}

\section{实验与结果}

我们对四种代表性的Python随机整数生成器进行了一次基准分析。对每一种,我们都在区间 $[2, 1,000,000]$ 内生成了5000个样本,并使用在相同区间上计算的基准进行检验。

\subsection{参赛者}
\begin{enumerate}
    \item \textbf{\texttt{random}}: Python的标准PRNG (梅森旋转算法)。
    \item \textbf{\texttt{secrets}}: Python的密码学安全RNG (CSPRNG)。
    \item \textbf{\texttt{numpy.random}}: NumPy库中的高性能PRNG。
    \item \textbf{简单LCG}: 一个教科书级的线性同余生成器,以其速度快但统计特性可能较弱而闻名。
\end{enumerate}

\subsection{结果}
验证结果总结在表 \ref{tab:results} 中。高质量的生成器——\texttt{random}, \texttt{secrets}, 和 \texttt{numpy.random}——都以很高的P值通过了所有测试,表明它们的输出在结构上与自然数分布无法区分。图 \ref{fig:results} 提供了视觉上的确认。

与此形成鲜明对比的是,简单LCG虽然通过了游程测试,但表现出明显的弱点。其K-S检验 (0.0522) 和特别是 $\chi^2$ 检验 (0.0265) 的P值,都可疑地接近于我们设定的显著性水平 $\alpha=0.01$,揭示了其结构输出中一个微小但系统性的偏差。这展示了 \texttt{WPF-Stat} 套件的灵敏度和诊断能力。

\begin{table}[h!]
    \centering
    \caption{不同Python RNG在\texttt{WPF-Stat}套件上的P值 ($\alpha=0.01$)}
    \label{tab:results}
    \begin{tabular}{@{}l c c c@{}}
        \toprule
        \textbf{RNG 库} & \textbf{K-S 检验} & \textbf{$\chi^2$ 检验} & \textbf{游程检验} \\
        \midrule
        \texttt{random}      & 0.0681            & 0.9435                 & 0.1837             \\
        \texttt{secrets}     & 0.1555            & 0.2099                 & 0.8430             \\
        \texttt{numpy.random}& 0.7279            & 0.1410                 & 0.6713             \\
        \texttt{Simple LCG}  & \textbf{0.0522}   & \textbf{0.0265}        & 0.3085             \\
        \bottomrule
    \end{tabular}
\end{table}

\begin{figure}[h!]
    \centering
    \includegraphics[width=0.49\textwidth]{benchmark_2_Crypto_Secrets.png}
    \includegraphics[width=0.49\textwidth]{benchmark_4_Simple_LCG.png}
    \caption{WPF熵分布比较。左:\texttt{secrets}库与自然基准表现出近乎完美的对齐。右:简单LCG与自然分布存在微小但可见的偏差,这些偏差被统计检验所捕获。}
    \label{fig:results}
\end{figure}

\section{讨论}

在我们初步实验中的一个关键发现是**熵的尺度效应**。一次将被测序列(来自 $[2, 10^6]$)与一个基准(来自 $[2, 10^5]$)进行比较的实验,导致了灾难性的、普遍的测试失败。这不是RNG的失败,而是 \texttt{WPF-Stat} 的一次正确探测,即这两个“数字宇宙”拥有完全不同的熵分布。更大的数有更高的概率包含更大的质因子,从而系统性地将整个WPF熵分布推向更高的值。这突显了为进行有效分析,匹配基准范围与测试范围的必要性。

\texttt{WPF-Stat} 在识别LCG的精细缺陷,同时确认主流库高质量方面的成功,验证了其作为一个正交工具的角色。它探测了随机性的一个全新维度——数论结构——而这对比特级分析是完全不可见的。

该方法的主要局限性在于其对整数分解的依赖,这对于密码学级别的大数在计算上是不可行的。因此,\texttt{WPF-Stat} 并非意图直接分析RSA的输出整数,而是用于审计为这类密码系统生成种子和素数的底层RNG。

\section{结论}

我们开发并验证了 \texttt{WPF-Stat},一个利用整数质因数分解中编码的结构信息的新颖随机性测试框架。我们的结果表明,它是一个灵敏而强大的工具,能够确认高质量RNG的结构完整性,同时成功识别出更原始生成器中深层次的缺陷。

通过提供一次对随机数序列结构健康的“CT扫描”,\texttt{WPF-Stat} 成为了在持续追求更高质量随机性的过程中的一个有价值的新工具。我们将此工具开源,希望能为研究社区提供一个新的视角和工具。未来的工作可能包括开发计算成本更低的WPF熵代理,以将此分析扩展到更大的数,以及探索该工具帮助我们可视化的、数轴上丰富的统计物理现象。

\section*{代码可用性}
本研究中描述的 \texttt{WPF-Stat} 工具的完整源代码、实验脚本和文档,均已作为开源项目发布在GitHub上,可访问以下链接获取:
\begin{center}
    \url{https://github.com/your-username/wpf-stat-project}
\end{center}
我们鼓励社区使用、验证和扩展这项工作。

\section*{参考文献}
\textit{(此部分为占位符,一篇正式的论文会在此处引用相关工作,例如:C.E. Shannon的《通信的数学理论》,D.E. Knuth的《计算机程序设计艺术》第二卷中关于伪随机数的章节,以及NIST SP 800-22文档等。)}

\appendix
\clearpage
\section{附录:核心代码实现}

\subsection{核心算法实现 (core.py)}
此模块包含了计算单个整数WPF熵的核心逻辑,并提供了并行计算的批处理功能。
\begin{lstlisting}
import math
from sympy.ntheory import factorint
from multiprocessing import Pool, cpu_count

def calculate_wpf(n: int) -> float:
    if n <= 1: return 0.0
    factors = factorint(n)
    if len(factors) == 1: return 0.0
    
    primes = list(factors.keys())
    exponents = list(factors.values())
    m = len(primes)
    A = sum(exponents)
    
    h_struct = -sum((k / A) * math.log2(k / A) for k in exponents)
    phi_order = 1 + math.log2(A)
    w_weight = sum(math.log2(p) for p in primes) / m
    
    return h_struct * phi_order * w_weight

def batch_calculate_wpf(numbers: list, parallel: bool = True):
    if not parallel:
        return [calculate_wpf(n) for n in numbers]
    with Pool(processes=cpu_count()) as pool:
        return pool.map(calculate_wpf, numbers)
\end{lstlisting}

\subsection{分析器类实现 (analyzer.py)}
这是工具的核心类,封装了基准加载、测试执行和结果可视化的全部功能。
\begin{lstlisting}
# analyzer.py (部分关键代码)
import numpy as np
from scipy import stats
import matplotlib.pyplot as plt
from .core import batch_calculate_wpf
from .benchmark import get_benchmark_dist

class WPFAnalyzer:
    def __init__(self, sequence: list):
        # ... 初始化代码 ...
        self.sequence_wpf = np.array(batch_calculate_wpf(self.sequence, parallel=True))
        self.results = []

    def test_entropy_distribution(self, alpha: float):
        # ... K-S 检验实现 ...
    
    def test_micro_structure(self, alpha: float):
        # ... Chi-Squared 检验实现 (包含合并低频次bin的逻辑) ...

    def test_median_runs(self, alpha: float):
        # ... 游程检验实现 ...

    def run_all_tests(self, alpha: float = 0.01):
        self.results = []
        self.test_entropy_distribution(alpha)
        self.test_micro_structure(alpha)
        self.test_median_runs(alpha)
        # ... 打印结果 ...

    def plot_distributions(self, filename: str, num_bins: int = 50):
        # ... Matplotlib 绘图代码 ...
\end{lstlisting}

\end{document}